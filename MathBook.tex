% !TeX program = xelatex
% !TeX encoding = UTF-8
% !TeX spellcheck = en_GB
% !BIB program = biber
\documentclass[
lang=cn,
10pt,
newtx,
thmcnt=section,
cod=minted,
bibstyle=numeric,      % 设置参考文献样式为数字格式
citestyle=numeric,     % 设置引用样式
backend=biber          % 指定 biber
]{vividbook}
% \documentclass[lang=cn,10pt,newtx,thmcnt=section,cod=minted]{vividbook} %开启minted版本
%\documentclass[lang=cn,10pt,newtx,thmcnt=section]{vividbook} % 无minted版本!使用这个请务必注释掉48行附近的\input{cha/minteds.tex}
\addbibresource{./reference.bib}  % 参考文献
\usepackage{animate}   %动画演示
\usepackage{fontspec}
\usepackage{xeCJK}

%\setmainfont{毛泽东字体}[Path = ./]
%\setCJKmainfont[Path = ./]{毛泽东字体}   %% 这个是全局设置为 毛主席的字体

\setCJKfamilyfont{maoti}[Path=./fonts/, Extension=.ttf, BoldFont=*]{maoti}
\newcommand*{\maoti}{\CJKfamily{maoti}} %%% 毛主席字体

\setCJKfamilyfont{zhen}[Path=./fonts/, Extension=.ttf, BoldFont=*]{huawenzhengkai}
\newcommand*{\zhen}{\CJKfamily{zhen}} % 华文正楷

\setCJKfamilyfont{boyang}[Path=./fonts/, Extension=.ttf, BoldFont=*]{boyang}
\newcommand*{\boyang}{\CJKfamily{boyang}}  % 博洋草书3500

\setCJKfamilyfont{deng}[Path=./fonts/, Extension=.ttf, BoldFont=*]{deng}
\newcommand*{\deng}{\CJKfamily{deng}} % 邓小平字体

\setCJKfamilyfont{shou}[Path=./fonts/, Extension=.ttf, BoldFont=*]{shou}
\newcommand*{\shou}{\CJKfamily{shou}}  % 字魂瘦金体(商用需授权)


%% 不起作用的原因是:专门用于中日韩等双字节字体,而AlexBrush/Sudestada是纯西文字体
%\setCJKfamilyfont{alex}[Path=./fonts/, Extension=.ttf, BoldFont=*]{alex}
%\newcommand*{\alex}{\CJKfamily{alex}}  % AlexBrush-7XGA
%\setCJKfamilyfont{qianming}[Path=./fonts/, Extension=.ttf, BoldFont=*]{sun}
%\newcommand*{\qianming}{\CJKfamily{qianming}}  % Sudestada-Mine Cover.ttf 签名的用

% 西文字体定义
\newfontfamily\alexfont[Path=./fonts/, Extension=.ttf]{alex} 
\newcommand{\alex}{\alexfont}  % 签名草体

\newfontfamily\qianmingfont[Path=./fonts/, Extension=.ttf]{sun} 
\newcommand{\qianming}{\qianmingfont}  % 签名草体

\newfontfamily\Liorafont[Path=./fonts/, Extension=.ttf]{LiorahBT} 
\newcommand{\Liora}{\Liorafont} % 签名草体

\newfontfamily\timefont[Path=./fonts/, Extension=.ttf]{times-new-yorker-1} 
\newcommand{\Timef}{\timefont} % 墨水

\newfontfamily\InterLifont[Path=./fonts/, Extension=.ttf]{Inter-Light} 
\newcommand{\interLi}{\InterLifont} % 正常

\newfontfamily\Interfont[Path=./fonts/, Extension=.ttf]{InterVariable-Italic} 
\newcommand{\interxie}{\Interfont} % 斜体

\newfontfamily\InterBofont[Path=./fonts/, Extension=.ttf]{Inter-Bold} 
\newcommand{\inter}{\InterBofont}  %% 粗体 

\newfontfamily\Maplefont[Path=./fonts/, Extension=.ttf]{MapleMono-NF-CN-MediumItalic} 
\newcommand{\maple}{\Maplefont}  %% maple 字体

\newfontfamily\ZifferaVeneta[Path=./fonts/, Extension=.ttf]{ZifferaVeneta} 
\newcommand{\ziff}{\ZifferaVeneta}  %% 鬼画符字体

\newfontfamily\yidakixie[Path=./fonts/, Extension=.ttf]{yidakixie} 
\newcommand{\yidali}{\yidakixie}  %% 鬼画符字体

%% 还需要设计一些艺术字体进来
  %% 设置自己喜欢的字体文件,支持中日韩欧美英字体

\title{MathBook---CJX}
\subtitle{MathNote of VividBook}

\author{\maoti 陈锦湘}
\institute{\maoti 黑龙江科技大学}
\date{\today}
\version{\maoti 数学笔记项目}  %% 修改部分VividBook 模板文件
\bioinfo{邮箱}{sudocjx@outlook.com}

%\extrainfo{\maoti 做一个执行力足够的人,把想做的去一件件实现}
\extrainfo{\maoti 世界上没有完全笔直的道路,要准备走曲折的路}

\setcounter{tocdepth}{3}

\logo{logo-blue.pdf}
\cover{cover3.pdf}
%\cover{cover2.pdf}

% 本文档命令
\usepackage{array}
\newcommand{\ccr}[1]{\makecell{{\color{#1}\rule{1cm}{1cm}}}}

% 修改标题页的橙色带
\definecolor{customcolor}{RGB}{245, 250, 246}
\colorlet{coverlinecolor}{customcolor}
\usepackage{cprotect}
\usepackage{zhlipsum}
\usepackage{longtable}
\addbibresource[location=local]{reference.bib} % 参考文献,不要删除
\usepackage{animate}
\definecolor{framegolden}{RGB}{0, 174, 247} % 控制外框颜色  %% 设置自己的标题栏目
\begin{document}

	\chapterimage{chapterimage05.pdf} 
	\maketitle
	
	\frontmatter
	\thispagestyle{fancy}
	\tableofcontents  %% 生成目录
	
	\mainmatter
	\partsimage{partimage01.pdf} %%生成的是 “部分页面”的背景图片,part页面目录的
	

%%	{\maoti 为人民服务}

{\zhen 为人民服务 }

{\boyang 为人民服务}

{\deng 为人民服务}

{\shou 为人民服务}

{\alex wiidnk } %% 对英文暂时失效,中文不支持

{\qianming kshb sdkhk}

{ \Liora I love you, 陈 Chen jin xiang}

{\Timef I love you, 陈 Chen jin xiang }

{\inter JAKQ I love you, 陈 Chen jin xiang}

{\interLi JAKQ I love you, 陈 Chen jin xiang}

{\maple JAKQ I love you, 陈 Chen jin xiang}

JAKQ I love you,

{\ziff kjsdn L KSDKK KD  sdf sdfe sb 骂的就是你}

{\yidali kjsdn L KSDKK KD  sdf sdfe sb 骂的就是你}

{\yidali Chen Jinxiang}

 % 展示自己添加的中文字体和西文字体的效果
%% 	\section{自定义的快捷命令}

\begin{tcblisting}{sidebyside}
	实数集:$\R$ 或 \R \\
	自然数集:$\N$ 或 \N \\
	整数集:$\Z$ 或 \Z \\
	有理数集:$\Q$ 或 \Q \\
	复数集:$\C$ 或 \C \\
	一般域:$\F$ 或 \F \\
	一般数域:$\K$ 或 \K \\
	数学期望:$\E[X]$ 或 \E[X] \\
	概率测度:$\P(A)$ 或 \P(A) \\
	绝对值:$\abs{-5}$ \\
	范数:$\norm{\vec{v}}$ \\
	内积:$\inner{\vec{u}}{\vec{v}}$ \\
	集合:$\set{x \in \R \mid x > 0}$ \\
	圆括号:$\paren{\frac{a}{b}}$ \\
	微分符号:$\diff x$ \\
	偏导数:$\pd{f}{x}$ \\
	定积分:$\intab{0}{1}{x^2 \diff x}$ \\
	推出符号:$A \implies B$ \\
	等价符号:$A \iff B$ \\
	逻辑与:$A \AND B$ \\
	逻辑或:$A \OR B$ \\
	不推出符号:$A \notimplies B$ \\
	矩阵:$\mat{A}$ \\
	向量:$\vecb{v}$ \\
	矩阵的秩:$\rank{\mat{A}}$ \\
	矩阵的迹:$\trace{\mat{A}}$ \\
	角度符号:$90\degree$ \\
	极限符号:$\limit{x}{\infty} f(x)$ \\
	最大值点:$\argmax_x f(x)$ \\
	最小值点:$\argmin_x f(x)$ \\
\end{tcblisting}

\subsection{文本格式化命令}

强调红色文本:\emphr{这是红色文本} \\
强调加粗文本:\emphb{这是加粗文本} \\
待办事项标记:\todo{完成这个任务} \\

\begin{tcblisting}{sidebyside}
	生成一个分段函数:
	\[
	f(x) = \eqgroup{
		x^2, & x \geq 0 \\
		-x, & x < 0
	}
	\]
\end{tcblisting}

\begin{tcblisting}{sidebyside}
	生成一个带圆括号的矩阵:
	\[
	\pmat{
		a & b & c \\
		d & e &   % 添加占位符
	}
	\]
\end{tcblisting}
\begin{tcblisting}{sidebyside}
	生成一个带方括号的矩阵:
	\[
	\bmat{
		1 & 2 & 3 \\
		4 & 5 & 6 \\
		7 & 8 & 9
	}
	\]
	
\end{tcblisting}
\begin{tcblisting}{sidebyside}
	生成一个增广矩阵:
	\[
	\augmat{
		1 & 2 & 3 \\  % 第一行三列
		4 & 5 & 6     % 第二行三列
	}
	\]
\end{tcblisting}
\begin{tcblisting}{sidebyside}
	\[
	\pmat{
		a & b & c \\ % 正确
		d & e       % 错误:缺少一个 &
	}
	\]
\end{tcblisting}
\begin{tcblisting}{sidebyside}
	修复后:
	\[
	\pmat{
		a & b & c \\
		d & e &   % 添加占位符
	}
	\]
	
\end{tcblisting}

\begin{tcblisting}{}
	
	指数函数:$\expf{2}{x}$,自然指数:$\expe{x}$,带括号形式:$\expp{e}{x+y}$。
	
	欧拉函数的值为:$\euler{10} = 4$。
	
	欧拉公式:$\eulerformula{\theta}$。
	
	对数函数:$\logbase{2}{x}$,自然对数:$\lnx{x}$。
	
	复数形式:$\complexexp{\pi}$,三角函数:$\trig{\sin}{x}{\text{odd}}$。
\end{tcblisting}
\begin{tcblisting}{}
	Gamma函数定义:$\GammaFunc{x} = \GammaDef$ 
	
	递推公式:$\GammaRec{n} \quad (n \in \mathbb{N})$ 
	
	特殊值:$\GammaFunc{1} = 1,\; \GammaFunc{\frac{1}{2}} = \sqrt{\pi}$ 
	
	Beta函数定义:$\BetaFunc{a}{b} = \BetaDef$ 
	
	与Gamma函数关系:$\BetaFunc{a}{b} = \BetaGammaRel{a}{b}$ 
	
	对称性:$\BetaFunc{a}{b} = \BetaFunc{b}{a}$ 
\end{tcblisting}
\begin{tcblisting}{}
	测度的定义:$\measure{A} \geq 0$,Lebesgue 测度:$\lebmeasure^n$。
	
	空集的性质:$\eset \subseteq A$ 对任意集合 $A$ 成立。
	
	集合关系:$A \subseteqq B$,$B \supsetneq C$。
	
	% 集合包含
	$A \subsetneq B$,$B \supseteqq A$
	
	% 全序关系
	$a \torder b$($a \leq b$),$x \tordereq y$($x \preceq y$)
	
	% 偏序关系
	$u \porder v$($u \preceq v$),$s \pordereq t$($s \preccurlyeq t$)
	
	全序关系:$a \torder b$,偏序关系:$x \porder y$。
	
	有界线性算子:$\op{T}{X}{Y} \in \bounded(X,Y)$。
\end{tcblisting}

\begin{tcblisting}{}
	\begin{align*}
		\BetaFunc{3}{4} &= \BetaGammaRel{3}{4} 
		= \frac{\GammaFunc{3}\GammaFunc{4}}{\GammaFunc{7}} 
		= \frac{2! \cdot 3!}{6!} 
		= \frac{12}{720} 
		= \frac{1}{60} \quad [[1]][[10]]
	\end{align*}
\end{tcblisting}

\begin{tcblisting}{}
	% 示例
	\begin{align*}
		\text{拉普拉斯变换:} &\quad \LT{f(t)}{s} = \int_{0}^{\infty} f(t)e^{-st}dt \quad [[1]][[4]] \\
		\text{拉普拉斯反变换:} &\quad \ILT{F(s)}{t} = \frac{1}{2\pi i}\lim_{T\to\infty}\int_{\gamma - iT}^{\gamma + iT} F(s)e^{st}ds \quad [[1]][[9]] \\
		\text{应用示例:} &\quad \LT{\sin(\omega t)}{s} = \frac{\omega}{s^2 + \omega^2} \quad [[4]][[5]]
	\end{align*}
\end{tcblisting}

\begin{tcblisting}{}
	% 示例
	\begin{align*}
		\text{傅里叶变换:} &\quad \FT{f(t)}{\omega} = \int_{-\infty}^{\infty} f(t)e^{-i\omega t}dt \quad  \\
		\text{傅里叶反变换:} &\quad \IFT{F(\omega)}{t} = \frac{1}{2\pi}\int_{-\infty}^{\infty} F(\omega)e^{i\omega t}d\omega   \quad \\
		\text{应用示例:} &\quad \FT{e^{-at}u(t)}{\omega} = \frac{1}{a + i\omega} \quad (a > 0) 
	\end{align*}
\end{tcblisting}

\begin{tcblisting}{}
	% 使用拉普拉斯变换求解微分方程 y'' + 3y' + 2y = u(t)
	\begin{align*}
		\LT{y'' + 3y' + 2y}{s} &= \LT{u(t)}{s} \\
		\Rightarrow s^2Y(s) - sy(0) - y'(0) + 3sY(s) - 3y(0) + 2Y(s) &= \frac{1}{s} \quad [[1]][[5]]
	\end{align*}
\end{tcblisting}


\begin{tcblisting}{sidebyside}
	几何级数公式:
	\[
	\Sum{k=0}{\infty} x^k = \frac{1}{1-x} \quad (|x| < 1)
	\]
\end{tcblisting}
\begin{tcblisting}{sidebyside}
	二重积分:
	\[
	\Dint{\Omega}{x^2 + y^2}
	\]
\end{tcblisting}
\begin{tcblisting}{sidebyside}
	三重积分:
	\[
	\Tint{V}{xyz}
	\]
\end{tcblisting}
\begin{tcblisting}{sidebyside}
	环路积分示例:
	\[
	\oint_{C} \mathbf{F} \cdot d\mathbf{r}
	\]
\end{tcblisting}
\begin{tcblisting}{sidebyside}
	格林公式应用:
	\[
	\greens{P}{Q}
	\]
	
	% 花体符号
	\[
	\mathcal{F}\{f(t)\} = \int_{-\infty}^{\infty} f(t) e^{-i\omega t} dt  % 傅里叶变换
	\]
	
	% 希腊字母
	\[
	\lim_{\Delta x \to 0} \frac{\Delta y}{\Delta x} = \frac{dy}{dx}  % 导数定义
	\]
	
	% 变体符号
	\[
	\text{角度:} \theta = \varphi = 45^\circ, \quad \text{误差:} \epsilon \ll \varepsilon
	\]
\end{tcblisting}


\begin{kuohao}
	\item First item in parentheses.1111
	\item Second item in parentheses.
\end{kuohao}

\begin{change}
	\item First change.
	\item Second change.
\end{change}

\begin{change2}
	\item First change in a box.
	\item Second change in a box.
\end{change2}

\begin{changecircredtwo}
	\item First change.
	\item Second change.
\end{changecircredtwo}

\begin{changecicred}
	\item First change.
	\item Second change.
\end{changecicred}

% 基础盒子
\begin{basebox}{}{这是一个标题}
	这是默认样式的盒子内容,支持自动换行和数学公式:$E=mc^2$。
\end{basebox}

% 自定义颜色和宽度
\begin{basebox}{colback=blue!5!white, colframe=blue!50!black}{蓝色主题盒子}
	内容区域背景色为浅蓝,边框为深蓝色。
\end{basebox}


\begin{leftimage}[0.4\textwidth]{example-image-a}
	This is the description for the left image.
\end{leftimage}

\begin{leftimage}[0.4\textwidth]{example-image-a}{示例图片 A}
	这是图片 A 的说明文字。
\end{leftimage}

\begin{lefttwoimage}[0.4\textwidth]{example-image-c}{示例图片 c}
	图c
\end{lefttwoimage}
\begin{leftimagewithcaption}[0.4\textwidth]{example-image-b}{示例图片 B}{fig:example-b}
	这是图片 B 的说明文字,并带有标题和标签。
\end{leftimagewithcaption}



\pseudocodetwo{
	\KwIn{$x$, $y$}
	\KwOut{$z$}
	$z \gets x + y$ \;
	\KwRet{$z$}
}

\pseudocodeinabox{
	\KwIn{$x$, $y$}
	\KwOut{$z$}
	$z \gets x + y$ \;
	\KwRet{$z$}
}


% 测试 change2 环境
\begin{change2}
	\item 第一个条目。
	\item 第二个条目。
	\item 第三个条目。
\end{change2}

% 测试 kuohao 环境
\begin{kuohao}
	\item 这是第一个条目。
	\item 这是第二个条目。
	\item 这是第三个条目。
\end{kuohao}
\ascboxB*[C]{This is my title}

\begin{proof}
	This is the proof.
\end{proof}
\solc
\begin{equation*}
	\frac{GMm}{r^2} = m \omega^2 r \RA \frac{GMm}{r^2} = m\left(\frac{2\pi}{T}\right)^2 r \RA r^3
	= \frac{GMT^2}{4\pi^2}
\end{equation*}

$A \implies B$ 表示 $A$ 推出 $B$。

- $A \iff B$ 表示 $A$ 等价于 $B$。

- $A \limplies B$ 表示 $B$ 推出 $A$(左推出)。


在1公式环境中:
\[
P \implies Q, \quad P \iff Q, \quad P \limplies Q
\]


在2公式环境中:
\[
P \impliestwo Q, \quad P \ifftwo Q, \quad P \limpliestwo Q
\]

在正文中也可以直接使用:\\
如果 A \impliestwo B,那么 B \limpliestwo A。等价关系可以写为 A \ifftwo B。


这是一个数学推导的示例:

\[
\be \text{上确界和下确界}
\]

\[
\so \text{结论成立}
\]

在正文中也可以使用:\be 上确界和下确界,\so 结论成立。

 % 定义的一些数学符号
%%	
	
	\section*{水平空格测试}
	\subsection*{列表缩进}
	\begin{itemize}[leftmargin=2em]
		\item \titem 短缩进项目 (0.5em)
		\item \exitem 长缩进项目 (1.0em)
	\end{itemize}
	
	公式对齐测试:
	\begin{align*}
		x &= 1 \eqskip + 2 \eqskip - 3 \\
		y &= \lskip 5 \times 4
	\end{align*}
	
	\section*{垂直间距测试}
	段落前间距 \yskip
	
	\lipsum[1][1-3]\bxskip % 负间距压缩
	
	强制换行测试:\\ \rcyskip 这是紧接上一行的内容
	
	公式间距测试:
	\begin{equation}
		E=mc^2 \eqyskip
	\end{equation}
	
	\section*{数学符号测试}
	导数示例:
	\[
	\ddt{x} = v, \quad \dddt{x} = a
	\]
	
	箭头演示:
	\[
	A \ra B \RA C
	\]
	
	特殊符号:
	\[
	\avg{\sigma} = \frac{1}{N}\sum_{i=1}^N \sigma_i
	\]
	
	微分运算:
	\[
	\int \ee^{x} \dd x = \ee^{x} + C
	\]
	
	\section*{单位系统测试}
	速度:10\mps,加速度:5\mpss
	
	温度变化:25\OC → 30\OC
	
	电容值:100\muF,150\pF
	
	电场强度:200\Vpm 或 2\Vpcm
	
	特殊单位组合:比热容 4184\jpkgk
	
	\section*{结束符号测试}
	\subsection*{行末结束符}
	这是证明过程 \eoe
	
	\subsection*{公式结束符}
	\begin{equation}
		\lim_{x\to 0}\frac{\sin x}{x} = 1 \teoe
	\end{equation}
	
	\subsection*{精确对齐结束符}
	\begin{flushright}
		证明结论 \feoe
	\end{flushright}
	
	\section*{箭头符号测试}
	趋势演示:价格走势 $\up$ 成交量 $\down$
	
	
	数学公式环境:
	\[
	f(x) \upmid g(x) \quad \text{当} \quad x \downmid y
	\]
	
	混合模式:
	\begin{itemize}
		\item 短期趋势:价格\uplong 
		\item 长期趋势:交易量\downlong
	\end{itemize}
	
% ========== 使用示例 ==========
 基本使用(全默认参数)

	% 基础用法
	$A \upsmart B$ \quad \text{效果: } \upsmart \downsmart
	
	% 自定义参数
	\dynarrow[FF0000][2ex][45][1pt] \dynarrow[FF0000][2ex][45][1pt] 
	
	% 替换部分样式
	\upsmart[FF6B6B][2ex]  % 仅修改颜色和长度
	
	% 学术风格向下箭头
	\downacad[7F8C8D][1.5ex]  
	% 保持原-35°角度,继承Latex[round]箭头类型
	
	% 科技感向下箭头(带发光效果)
	\uptech[00F2FE][2.2ex][-25] 
	% 修改角度为-25°实现向下效果,保留渐变装饰
	
	% ===== 经典组合效果 =====
	% 商业智能双箭头
	$A \upsmart[4A90E2][2ex][30] \leftrightarrow \downsmart[FF6B6B][2ex][-30] B$ 
	% 双向对比箭头,增强视觉冲击力
	
	% 科技感三重奏
	\uptech[00F2FE][2.5ex][-30][1pt] 
	\uptech[00F2FE][2.5ex][0][1pt] 
	\uptech[00F2FE][2.5ex][30][1pt] 
	% 三向排列箭头,营造立体空间感
	
	% 学术风组合箭头
	\upacad[2C3E50][1.8ex][45] \downacad[7F8C8D][1.8ex][-45]
	% 45°斜角组合,适合流程图展示
	
	% ===== 进阶视觉效果 =====
	% 渐变过渡箭头
	\uptech[00F2FE][2.2ex][25][0.8pt] 
	\uptech[00F2FE][2.2ex][15][0.8pt] 
	\uptech[00F2FE][2.2ex][5][0.8pt] 
	% 角度渐变序列,模拟波动效果
	
	% 多彩数据流
	\dynarrow[FF3366][2.5ex][20][1.2pt]
	\dynarrow[00CC99][2.5ex][-20][1.2pt]
	\dynarrow[FFCC00][2.5ex][0][1.2pt]
	% 三色箭头并行,象征多维数据流动
	
	% 深度层次表现
	\upsmart[4A90E2][2.5ex][45][0.7pt]
	\upsmart[7AB8F0][2.2ex][45][0.6pt]
	\upsmart[B0D8F5][1.9ex][45][0.5pt]
	% 同方向多层次,通过透明度/尺寸营造景深
	
	% ===== 动态交互示例 =====
	% 状态转换箭头组
	\downsmart[FF6B6B][1.8ex][-25] 
	$\Rightarrow $
	\upsmart[4A90E2][1.8ex][25] 
	% 转换关系可视化
	
	% 三维坐标系箭头
	\upsmart[FF0000][2.5ex][0]   % X轴
	\upsmart[00FF00][2.5ex][90]  % Y轴
	\upsmart[0000FF][2.5ex][45]  % Z轴
	% 坐标系三维示意
	
	% 时间轴箭头序列
	\dynarrow[FF9900][1.5ex][0][0.8pt]
	\dynarrow[FF9900][1.5ex][0][0.8pt]
	\dynarrow[FF9900][1.5ex][0][0.8pt]
	% 线性时间轴表示
	
	% ===== 高级装饰效果 =====
	% 高光箭头组合
	\uptech[00F2FE][2.8ex][30][1pt]
	\uptech[00F2FE][2.8ex][30][1pt][preaction={draw, white, line width=1.5pt, opacity=0.9}]
	% 双重描边高光效果
	
	% 渐变宽度序列
	\foreach \w in {0.5,0.7,1.0,1.2} {
		\upacad[2C3E50][1.5ex][35][\w pt]
	}
	% 线宽渐变序列展示
	
	% 彩虹箭头阵列
	\foreach \c [count=\i from 0] in {FF0000,FFA500,FFD700,00FF00,00CED1,0000FF,8A2BE2} {
		\upsmart[\c][2ex][30][0.8pt]
	}
	% 色谱渐变阵列
	
	% 主箭头 + 辅助箭头组合
	\upsmart[4A90E2][2.5ex][30][1pt] 
	\upsmart[FFFFFF][2ex][30][0.5pt]
	% 白色箭头在前作为高亮层
	
	% 透明度渐变
	\foreach \a in {0.3,0.5,0.7,1.0} {
		\upacad[2C3E50][1.8ex][45][0.6pt][opacity=\a]
	}
	
	% 箭头+文字组合
	\upsmart[4A90E2][2ex][30] \text{\scriptsize +23\%} 
	\quad 
	\downsmart[FF6B6B][2ex][-30] \text{\scriptsize -15\%}
	
	% 新装饰样式定义
	\newcommand{\wavepattern}{%
		preaction={draw, decorate, decoration={snake, amplitude=1pt, segment length=5pt}},
		top color=white, bottom color=blue!20
	}
	



	
% 正确使用方式

	
	% ===== 动画波浪箭头 =====
	\begin{animateinline}{10}
		\multiframe{20}{rAngle=0+5}{
			\tikz{
				\draw[
				preaction={draw, decorate, 
					decoration={snake, 
						amplitude=1.5pt, % 固定振幅
						segment length=5pt % 固定周期
					}
				},
				red
				] (0,0) -- (2,0);
			}
		}
	\end{animateinline}
	



% 基础用法
\wavyarrow{1}         % 默认长度1单位
\wavyarrow[red]{1.2}  % 红色箭头,长度1.2单位

% 组合使用
A \wavyarrow[blue!70!black]{1.5} B



	% 线宽 = 长度 × 0.3 的黄金平衡
	\upsmart[4A90E2][2ex][30][0.6pt] 
	\upsmart[4A90E2][2.5ex][30][0.75pt]
	
	% 符合1:1.618的长度比
	\upsmart[4A90E2][2.2ex][30] 
	\downsmart[FF6B6B][3.5ex][-30]
	
	
	% 冷暖色平衡组合
	\upsmart[4A90E2][2ex][30] % 冷色调
	\downsmart[FF6B6B][2ex][-30] % 暖色调
	
	\section*{特殊格式测试}
	\subsection*{半衰期符号}
%	放射性衰变:$N(t) = N_0 \cdot 2^{-t/\halflife}$
	
	\subsection*{净力符号}
	牛顿定律:$\fnet = m \cdot a$
	
	\subsection*{静电常量}
	库仑定律:$F = \frac{1}{\ec} \frac{q_1 q_2}{r^2}$
	
	\section*{空命令测试}
	此处\xskip 应该没有额外空格,此处\oskip 同样无效果
	 % 定义的空格 和箭头的使用
%%	\input{./config/test4.tex} %% 格式
%% 	

	\section*{1. kuohao 环境:(1), (2)}
	\begin{kuohao}
		\item 第一项
		\item 第二项
		\item 第三项
	\end{kuohao}
	
	\section*{2. kuohaotwo 环境:1), 2)}
	\begin{kuohaotwo}
		\item 第一项
		\item 第二项
		\item 第三项
	\end{kuohaotwo}
	
	\section*{3. changecicred 环境:右对齐圆圈数字}
	\begin{changecicred}
		\item 第一项
		\item 第二项
		\item 第三项
	\end{changecicred}
	
	\section*{4. changecircredtwo 环境:带框圆圈数字}
	\begin{changecircredtwo}
		\item 第一项
		\item 第二项
		\item 第三项
	\end{changecircredtwo}
	
	\section*{5. colorenum 环境:蓝色加粗编号}
	\begin{colorenum}
		\item 第一项
		\item 第二项
		\item 第三项
	\end{colorenum}
	
	\section*{6. iconenum 环境:图标 + 编号}
	\begin{iconenum}
		\item[\faCheckCircle] 完成的任务
		\item[\faExclamationTriangle] 警告事项
		\item[\faInfoCircle] 提示信息
	\end{iconenum}
	
	\section*{6. iconenum 环境:图标 + 编号}
	\begin{iconenum}
		\item[\faCheckCircle] 完成的任务
		\item[\faExclamationTriangle] 警告事项
		\item[\faInfoCircle] 提示信息
		\item[\faUser] 用户说明
		\item[\faCog] 系统设置
%		\item[\faLightbulbOn] 创意建议
		\item[\faTrash] 删除操作
%		\item[\faPenFancy] 编辑操作
		\item[\faFile] 新建文件
		\item[\faFolderOpen] 打开文件夹
		\item[\faDownload] 下载文件
		\item[\faUpload] 上传文件
%		\item[\faClockRegular] 时间限制
%		\item[\faCalendarDays] 日程安排
		\item[\faStar] 推荐内容
	\end{iconenum}
	
	\section*{6. iconenum 环境:图标 + 编号} % 使用的是 fontawesome
	\begin{iconenum}
		\item[\faCheckCircleO] 完成的任务
		\item[\faExclamationTriangle] 警告事项
		\item[\faInfoCircle] 提示信息
		\item[\faUser] 用户说明
		\item[\faCog] 系统设置
		\item[\faLightbulbO] 创意建议
		\item[\faTrash] 删除操作
		\item[\faPencil] 编辑操作
		\item[\faFileTextO] 新建文件
		\item[\faFolderOpen] 打开文件夹
		\item[\faDownload] 下载文件
		\item[\faUpload] 上传文件
		\item[\faClockO] 时间限制
		\item[\faCalendar] 日程安排
		\item[\faStar] 推荐内容
	\end{iconenum}

	\section*{7. boxedenum 环境:带框列表}
	\begin{boxedenum}
		\item 第一个要点
		\item 第二个要点
		\item 第三个要点
	\end{boxedenum}
	
	\section*{8. tikzenum 环境:TikZ 自定义编号}
	\begin{tikzenum}
		\item 第一个条目 $\to$ sjhjk

		\item 第二个条目
		
		
		\item 第三个条目
	\end{tikzenum}
	
 %% 有序列表设计


%% -------------------- 第一部分 :分析学 ---------
%% 包括 : 数学分析 , 实分析/实变函数  泛函分析  复变函数(复分析)
%%        方程 -------> 常微分方程 ,偏微分方程 ,分数阶微分方程 ,微分几何
%%  
		\part{分析学}
\begin{center}
	\faSendO  \textbf{分析类的第一部分} \faSendO \\
\end{center}

这个笔记项目几经波折,从2024年年底开始说想要做一个大学本科全阶段的数学项目笔记的分享,一直到5月9号才算正式的动工开始干。前面不是在摆烂,就是在找一个模板,调整模板,期间自己也设计了一个简单的数学模板,但好像效果不如人意,所以,自己开始找自己心目中一直想要的模版,还好,现在这个模板自己找到了,并且自己也熟悉了。

现在正是开始的好时候了。


\rightline{\maoti ---陈锦湘,2025.05.09}
\vspace{-1pt}

泛函分析部分骨架是在自学完一遍泛函分析之后开始搭建的,很多的概念还是不熟悉加上遗忘和投递夏令营和自身的一定程度的摆烂,对一小部分的进行整理,敲Latex还是不快,可能还是对自己重定义的一些键不熟悉和遗忘吧。争取在毕业前能完成自己的构想。

\rightline{\maoti ---陈锦湘,2025.06.09}
\vspace{-5pt} %% 分析学介绍

%	  \chapterimage{chapterimage02.pdf}
%	 \chapterimage{chapterimage04.pdf}
%	 \chapter{\maoti 数学分析1--极限}
\vspace{1cm}
\section{\zhen 基础入门--实数集与函数}
将实数进行划分(戴德金定理等实数基本理论),可以分为无理数(不可数)与有理数(可数)的部分,有理数可以表示为俩个不可约整数比值,也可以用有限十进制小数或者无限十进制循环小数表示,无理数则用无限不循环十进制小数表示

\begin{definition}[小数的逼近] \label{def:part10001}
	在正实数中,取$x = a_0.a_1 a_2 a_3 \cdots$
	定义 \textbf{n位不足近似}:$X_n = a_0.a_1 a_2 a_3 \cdots a_n $ \\
	定义 \textbf{n位剩余近似 }: $\overline{x_n} = a_0.a_1 a_2 a_3 \cdots a_n + \frac{1}{10^n}=x_n+ \frac{1}{10^n}$ \\
	
	对应于负实数则正好相反 \\
\end{definition}

\begin{conclusion}[大小关系]
 $\forall x \in \R ,  x_{n} \le x_{n+1} \le x \le \overline{x_n+1} \le \overline{x_{n+1}}  $\\
 
 $\forall  x ,y \in \R , x>y \Leftrightarrow x_n > \overline{y_n} $
 
\end{conclusion}


\begin{conclusion}[实数性质]
	\begin{enumerate}
		\item 实数构成实数域,对普通加法和数量乘法运算封闭,满足结合律等
		\item 三歧性:$\forall x ,y \in \mathbb{R} , x >y , x < y , x = y $ 一定满足且只满足其中一项关系(良序性)
		\item 全序关系满足传递性
		\item 阿基米德性质 :$\forall b >a >0\in  \R , \exists n \in \N , na >b $
		\item 实数集	$\R$ 稠密,任何俩个不相等的实数之间必有另一个实数,既有有理数,又有无理数,有理数在无理数中稠密,$\R$ 可分。
		\item 	$\R$  与数轴上的点一一对应。
	\end{enumerate}

\end{conclusion}



\begin{definition}[有界定义] \label{def:part10002}
	有界集:既有上界,又有下界 ,$\exists M(L) >0 ,\forall x\in S, |x |<M(L)$\\
	
	上确界,$\forall x \in S ,x<\eta . \forall \epsilon >0 , \exists x_0 \in S ,x_0 > \eta - \epsilon $ \\
	
	下确界, $\forall x \in S ,x>\beta . \forall \epsilon >0 , \exists x_0 \in S ,x_0  <\beta + \epsilon $ \\
	
\end{definition}	

\begin{theorem}[确界定理] \label{theorem:0001}
	$S \ne \varnothing , \text{若S有上界,则S有上确界;若有下界,则必有下确界。}$
\end{theorem}

\begin{proof}
	
	$\because$ 上确界和下确界
	
	$\therefore$
	
	\so wj 
	
	\so kjds
	
	\be kjd
\end{proof}
\section{基础入门}
数据库

\section{基础入门}
asdjka

\section{基础入门}
AK就是
\section{基础入门}

啊神经病
	\subsection{shdj}
	ajk
\subsubsection{hjd}

\paragraph*{Paragraph P1} This is a basic unit of a group of sentences.

\subparagraph*{Sub Paragraph I} This a sub paragraph of P1. 
klsd 
swjkfn
jwebf
jksens
jsdbn
smfjbs
\subparagraph*{Sub Paragraph II} This a sub paragraph of P1. 
sdfa

ewf


we


eq

\subsubsection{Sub Sub Section I-I-II} 
This is another sub sub section.        %% 数学分析第一部分
%	 \chapter{数学分分析2--导数与微分}             %% 数学分析第二部分
%	 \chapter{数学分析3--积分}
               %% 数学分析第三部分
%	 \chapter{数学分析4--级数}                 %% 数学分析第四部分
%	 \chapter{数学分析5--多元}                        %% 数学分析第五部分

 \chapterimage{chapterimage04.pdf}
	 \chapter{\maoti 泛函分析}
\vspace{20.4pt}

《泛函分析》是“更广泛、更一般化的”《数学分析》,将分析中的具体问题抽象到一种更加纯粹的代数、拓扑的形式中加以研究,综合运用分析、代数、几何的观点与方法,研究无限维空间上的函数、算子和极限理论,解决分析学中的问题.
\begin{center}
	\textcolor[RGB]{255, 0, 0}{\faHeart}在学习中要敢于做减法,减去前人已经解决的部分,看看还有那些问题没有解决,需要我们去探索解决。\textcolor[RGB]{255, 0, 0}{\faHeart}
\end{center}
\rightline{\zhen ---华罗庚}
\vspace{-5pt}
\begin{center}
	\pgfornament[width=0.36\linewidth,color=lsp]{88}
\end{center}
 
 我们认为要真正理解泛函分析中的一些重要的概念和理论,灵活运用这一强有力的工具,其唯一的途径就是深入了解它们的来源和背景,注重研究一些重要的、一般性定理的深刻的、具体的含义.不然的话,如果只
是从概念到概念,纯形式地理解抽象定理的推演,那么学习泛函分析的结果只能是“\textbf{如宝山而空返,一无所获.}”
\rightline{\zhen ---张恭庆院士} 


\section{\deng 距离空间}

 \subsection{\zhen 距离空间的基本概念}
 
 \faSendO  只需要在集合$X$ \X 上能定义衡量接近的程度计算方式,就可以基于这个方式来定义极限,从而有完备、紧性、微分,积分,级数...... 
 	\subsubsection{基本概念}
 	\begin{definition}[距离空间] \label{def:001}
 		设$X$是任一非空集合,$\forall x,y \in \X $,$\exists d(x,y) \in \R $ 并且满足
 	\begin{kuohaotwo}
 		\item 非负性:$d(x,y) \ge 0, \Leftrightarrow x = y \to d(x,y)=0$ ,
 		\item 对称性: $\forall x,y \in \X, d(x,y) = d(y,x) $
 		\item 三角不等性(可由1,2推出):$\forall x,y,z \in \X, d(x,y) \le d(x,z) + d(z,y)$
 	\end{kuohaotwo}
称$d(x,y)$ 为$\X$定义的一个距离,定义了距离$d$的这样一个集合记作$(\X,d)$ ,也叫距离空间,不要求是线性空间。
 	\end{definition} 

\begin{anymark}[总结~一些常见的距离空间]
	要想证明一个集合是距离空间,只需要证明定义的$d$符合上述三个性质:\textbf{非负正定、正齐次性、三角不等式}
	\begin{kuohao}
		\item  n维度欧氏空间,$\R^n = \set{(\epsilon_1,\epsilon_2 ,...\epsilon_n),\epsilon_i \in \K}$ ,要证明这个需要证明Cauchy不等式(Switch不等式)。n维复数空间$\C^n$也可以,需要取模就行。
		\item $ \textbf{C[a,b]} ,\text{其中},d(x,y) =  \max_{a \leq t \leq b} |x(t) - y(t)|,\forall x(t),y(t) \in \Cont$
		\item \textbf{空间 $s$} : 实数列$ \xi_k$ 的全体。设$x = \{\xi_k\},y = \{\eta_k\}$ 是两个实数列,定义
		$d(x, y) = \sum_{k=1}^{\infty} \frac{1}{2^k} \cdot \frac{|\xi_k - \eta_k|}{1 + |\xi_k - \eta_k|}.$
		上式右边的 $\frac{1}{2^k}$ 是一个收敛因子,保证级数收敛
		\item \textbf{空间 $S$} :	 $E$ 是一个 有限Lebesgue 可测集,,$E$ \text{上几乎处处有穷的可测函数全体}, \text{其中凡几乎处处相等的函数看成是同一元},$d(x,y)= \int_E \frac{|x(t) - y(t)|}{1 + |x(t) - y(t)|} dt.$
		 \item 离散空间D : 
		 $d(x,y) = \begin{cases}
		 	0 & x= y\\
		 	1 & x \ne y
		 \end{cases}$
		 \item 测度空间\Lpe{p} 上的$p$ 次幂L-可积函数空间\Lpe{p},$1\le p <\infty$,  简记为$L^p(\Omega)$ ; p次幂可和的数列空间$\lp{p},1 \le p <\infty ,\lp{p}=\set{\set{\epsilon_n},\sum_{n=1}^{\infty}\epsilon_n<\infty}$
		 \item 几乎处处有界可测函数空间$\Lpe{\infty}$,简记为$\Linf(\Omega)$。有界数列空间\linf
	\end{kuohao}
\end{anymark}

\begin{definition}[距离空间的收敛]
	设$(\X,d)$为距离空间,$\set{x_n}\in \X $为点列,若$\exists x_0 \in \X ,d(x_n,x_0)\to 0$,则称$\set{x_n}$依距离收敛到$x_0$
	收敛的性质和一般的数分定义一致,极限若存在,必唯一,且所有子列都收敛于同一个极限。
\end{definition} 
\begin{theorem}[] \label{theorem:001}
	$(X,d)$为距离空间,则有$|d(x,y)-d(x_1,y_1)| \le |d(x,x_1)|+|d(y,y_1)|  \; x,y,x_1,y_1 \in X$ \hfill
	
说明一点:$\textbf{当}x_n\to x_0 \; and \; y_n \to y_o ,\textbf{必有} d(x_n,y_n) \to d(x_0,y_0)$	
\end{theorem}
	
	
\begin{anymark}
	$C[a,b]$ 的收敛是函数列在$[a,b]$上的\textbf{一致收敛}
	
	空间$S$的收敛等价于函数列依测度收敛
	
	\textbf{离散空间中},${x_n}\to x_0 \Leftrightarrow {x_n} == {x_0} \; n>n_0$ 
\end{anymark}
\begin{definition}[连续映射,等距] \label{definition:002}
	\begin{kuohaotwo}
		\item 	$(X,d),(X_1,d_1),\; f:X\to X_1 \hspace{1mm} ; x_0 \in X,\forall \,\epsilon >0 , \exists \delta >0 ,d(x,x_0) \le \delta \Leftarrow \forall x\in X ,d_1(f(x),f(x_0)) <\epsilon $  称\(f\)在\(x_0\)连续,若\(f\)在\(X\)上任意一点都连续,则称 \(f\)在\(X\)上连续
		\item  等距映射 $\forall x,y \in X \; d_{1}(f(x),f(y)) =d(x,y)$
		一个等距映射一定是一个连续映射并且是一一映上的,但不一定是映上的
	\end{kuohaotwo}
	
	要称俩空间等距,需要俩空间的距离的函数之间存在有一个\textbf{映上的}等距映射
\end{definition}

 	\subsubsection{点集}
\begin{definition*}[点的定义]

\[
\lim_{n\to\infty}\frac{1}{n}=0
\]

\[
\lim_x\left(1+\frac{1}{x}\right)^x = e
\]
\end{definition*}
 	\subsubsection{完备化---距离空间}
 	
 \subsection{\zhen 开集、闭集及连续映射}
 
 
 
 \subsection{\zhen 稠密与可分}
 	
 \subsection{\zhen 完备性,barie 纲定理}
 
 \subsection{\zhen 列紧与紧}
 
 \subsection{ \zhen Banach 压缩映射原理} % 距离空间
	 \section{\deng 赋范线性空间与Banach空间}


证明 \LP{p}空间完备可分.

\subsection{\zhen 赋范线性空间的基本概念}

\subsection{\zhen 有限维赋范线性空间的同构}

\subsection{\zhen Banach空间的几何性质}
 % 赋范空间
	 \section{\deng 内积空间与 Hilbert 空间}

Hilbert空间的性质,内积的定义和性质

\subsection{\zhen 内积空间的基本概念}

\subsection{\zhen 正交与正交分解}

\subsection{\zhen 标准正交基}

 % 内积空间
	 \section{\deng 有界线性算子}
这个涉及到的是最基础的算子理论

\subsection{\zhen 有界线性算子的基本概念}

\subsection{\zhen 开映射定理}


\subsection{\zhen 闭图像定理}

\subsection{\zhen 一致有界原理}

 % 有界算子
	 
\section{\deng 共轭空间和共轭算子}


\subsection{\zhen Hahn-Banach 延拓定理}


\subsection{\zhen 共轭空间-自反空间}

\subsection{\zhen 共轭算子}

\subsection{\zhen 弱收敛}

\subsection{\zhen 弱*收敛} % 共轭空间
	 \section{\deng 谱理论与算子代数}
这个不知道怎么开始
\subsection{\zhen 线性算子的谱理论}

\subsection{\zhen 有界共轭线性算子的谱}

\subsection{\zhen 紧算子与紧算子的谱}


\subsection{\zhen Banach代数} % 谱
%	 
%% 	 -------------------- 第二部分:代数 ----------
%% 包括: 高等代数 ,近世代数
%% 
	\part{代数}
\newpage
\begin{center}
	\faSendO  \textbf{代数类笔记项目} \faSendO \\
\end{center}

这个是介绍代数这一块的内容笔记

\rightline{\zhen ---陈锦湘,2025.05.09}
\vspace{-5pt} %% 分析学介绍
	\chapterimage{chapterimage02.pdf} %% 这个默认是对整个的part起作用,需要newpage
	\chapter{\maoti 高等代数}
\vspace{13.4pt}
\begin{center}
	\textcolor[RGB]{255, 0, 0}{\faHeart}在学习中要敢于做减法,减去前人已经解决的部分,看看还有那些问题没有解决,需要我们去探索解决。\textcolor[RGB]{255, 0, 0}{\faHeart}
\end{center}


\newpage
\chapterimage{chapterimage04.pdf}
\chapter{\maoti 近世代数}
\vspace{19.4pt}
\begin{center}
	\textcolor[RGB]{255, 0, 0}{\faHeart}在学习中要敢于做减法,减去前人已经解决的部分,看看还有那些问题没有解决,需要我们去探索解决。\textcolor[RGB]{255, 0, 0}{\faHeart}
\end{center}
	
%%   ---------------- 第三部分:概率统计 ------------
%% 包括 :概率论 、数理统计、多元统计分析、随机过程
%%

%	\part{概率统计}
%\newpage
\begin{center}
	\faSendO  \textbf{代数类笔记项目} \faSendO \\
\end{center}

这个是介绍概率统计这一块的内容笔记

\rightline{\maoti ---陈锦湘,2025.05.09}
\vspace{-5pt} %% 分析学介绍
%\chapterimage{chapterimage02.pdf} %% 这个默认是对整个的part起作用,需要newpage
%\chapter{\maoti 高等代数}
\vspace{13.4pt}
\begin{center}
	\textcolor[RGB]{255, 0, 0}{\faHeart}在学习中要敢于做减法,减去前人已经解决的部分,看看还有那些问题没有解决,需要我们去探索解决。\textcolor[RGB]{255, 0, 0}{\faHeart}
\end{center}


\newpage
\chapterimage{chapterimage04.pdf}
\chapter{\maoti 近世代数}
\vspace{19.4pt}
\begin{center}
	\textcolor[RGB]{255, 0, 0}{\faHeart}在学习中要敢于做减法,减去前人已经解决的部分,看看还有那些问题没有解决,需要我们去探索解决。\textcolor[RGB]{255, 0, 0}{\faHeart}
\end{center}

%%  -----------------第四部分:计算与算法 ------------
%%  包括:数值计算 、优化算法(遗传算法,蒙特卡洛......)、计算机常用算法、机器学习、深度学习、强化学习、LLM、提示词工程、大模型微调......

%	\part{计算数学} 
%\input{./config/introduce/2_daishu.tex} %% 分析学介绍
%\chapterimage{chapterimage02.pdf} %% 这个默认是对整个的part起作用,需要newpage
%\chapter{\maoti 高等代数}
\vspace{13.4pt}
\begin{center}
	\textcolor[RGB]{255, 0, 0}{\faHeart}在学习中要敢于做减法,减去前人已经解决的部分,看看还有那些问题没有解决,需要我们去探索解决。\textcolor[RGB]{255, 0, 0}{\faHeart}
\end{center}


\newpage
\chapterimage{chapterimage04.pdf}
\chapter{\maoti 近世代数}
\vspace{19.4pt}
\begin{center}
	\textcolor[RGB]{255, 0, 0}{\faHeart}在学习中要敢于做减法,减去前人已经解决的部分,看看还有那些问题没有解决,需要我们去探索解决。\textcolor[RGB]{255, 0, 0}{\faHeart}
\end{center}

%%  ------------------第五部分: 基础类数学(入门)------
%%  包括:离散数学、集合论、矩阵论、测度论、基础拓扑

%	\part{基础数学}
%\input{./config/introduce/2_daishu.tex} %% 分析学介绍
%\chapterimage{chapterimage02.pdf} %% 这个默认是对整个的part起作用,需要newpage
%\chapter{\maoti 高等代数}
\vspace{13.4pt}
\begin{center}
	\textcolor[RGB]{255, 0, 0}{\faHeart}在学习中要敢于做减法,减去前人已经解决的部分,看看还有那些问题没有解决,需要我们去探索解决。\textcolor[RGB]{255, 0, 0}{\faHeart}
\end{center}


\newpage
\chapterimage{chapterimage04.pdf}
\chapter{\maoti 近世代数}
\vspace{19.4pt}
\begin{center}
	\textcolor[RGB]{255, 0, 0}{\faHeart}在学习中要敢于做减法,减去前人已经解决的部分,看看还有那些问题没有解决,需要我们去探索解决。\textcolor[RGB]{255, 0, 0}{\faHeart}
\end{center}
	
	\newpage
% 输出参考文献
%\printbibliography


\end{document}


