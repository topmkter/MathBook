\section{自定义的快捷命令}

\begin{tcblisting}{sidebyside}
	实数集:$\R$ 或 \R \\
	自然数集:$\N$ 或 \N \\
	整数集:$\Z$ 或 \Z \\
	有理数集:$\Q$ 或 \Q \\
	复数集:$\C$ 或 \C \\
	一般域:$\F$ 或 \F \\
	一般数域:$\K$ 或 \K \\
	数学期望:$\E[X]$ 或 \E[X] \\
	概率测度:$\P(A)$ 或 \P(A) \\
	绝对值:$\abs{-5}$ \\
	范数:$\norm{\vec{v}}$ \\
	内积:$\inner{\vec{u}}{\vec{v}}$ \\
	集合:$\set{x \in \R \mid x > 0}$ \\
	圆括号:$\paren{\frac{a}{b}}$ \\
	微分符号:$\diff x$ \\
	偏导数:$\pd{f}{x}$ \\
	定积分:$\intab{0}{1}{x^2 \diff x}$ \\
	推出符号:$A \implies B$ \\
	等价符号:$A \iff B$ \\
	逻辑与:$A \AND B$ \\
	逻辑或:$A \OR B$ \\
	不推出符号:$A \notimplies B$ \\
	矩阵:$\mat{A}$ \\
	向量:$\vecb{v}$ \\
	矩阵的秩:$\rank{\mat{A}}$ \\
	矩阵的迹:$\trace{\mat{A}}$ \\
	角度符号:$90\degree$ \\
	极限符号:$\limit{x}{\infty} f(x)$ \\
	最大值点:$\argmax_x f(x)$ \\
	最小值点:$\argmin_x f(x)$ \\
\end{tcblisting}

\subsection{文本格式化命令}

强调红色文本:\emphr{这是红色文本} \\
强调加粗文本:\emphb{这是加粗文本} \\
待办事项标记:\todo{完成这个任务} \\

\begin{tcblisting}{sidebyside}
	生成一个分段函数:
	\[
	f(x) = \eqgroup{
		x^2, & x \geq 0 \\
		-x, & x < 0
	}
	\]
\end{tcblisting}

\begin{tcblisting}{sidebyside}
	生成一个带圆括号的矩阵:
	\[
	\pmat{
		a & b & c \\
		d & e &   % 添加占位符
	}
	\]
\end{tcblisting}
\begin{tcblisting}{sidebyside}
	生成一个带方括号的矩阵:
	\[
	\bmat{
		1 & 2 & 3 \\
		4 & 5 & 6 \\
		7 & 8 & 9
	}
	\]
	
\end{tcblisting}
\begin{tcblisting}{sidebyside}
	生成一个增广矩阵:
	\[
	\augmat{
		1 & 2 & 3 \\  % 第一行三列
		4 & 5 & 6     % 第二行三列
	}
	\]
\end{tcblisting}
\begin{tcblisting}{sidebyside}
	\[
	\pmat{
		a & b & c \\ % 正确
		d & e       % 错误:缺少一个 &
	}
	\]
\end{tcblisting}
\begin{tcblisting}{sidebyside}
	修复后:
	\[
	\pmat{
		a & b & c \\
		d & e &   % 添加占位符
	}
	\]
	
\end{tcblisting}

\begin{tcblisting}{}
	
	指数函数:$\expf{2}{x}$,自然指数:$\expe{x}$,带括号形式:$\expp{e}{x+y}$。
	
	欧拉函数的值为:$\euler{10} = 4$。
	
	欧拉公式:$\eulerformula{\theta}$。
	
	对数函数:$\logbase{2}{x}$,自然对数:$\lnx{x}$。
	
	复数形式:$\complexexp{\pi}$,三角函数:$\trig{\sin}{x}{\text{odd}}$。
\end{tcblisting}
\begin{tcblisting}{}
	Gamma函数定义:$\GammaFunc{x} = \GammaDef$ 
	
	递推公式:$\GammaRec{n} \quad (n \in \mathbb{N})$ 
	
	特殊值:$\GammaFunc{1} = 1,\; \GammaFunc{\frac{1}{2}} = \sqrt{\pi}$ 
	
	Beta函数定义:$\BetaFunc{a}{b} = \BetaDef$ 
	
	与Gamma函数关系:$\BetaFunc{a}{b} = \BetaGammaRel{a}{b}$ 
	
	对称性:$\BetaFunc{a}{b} = \BetaFunc{b}{a}$ 
\end{tcblisting}
\begin{tcblisting}{}
	测度的定义:$\measure{A} \geq 0$,Lebesgue 测度:$\lebmeasure^n$。
	
	空集的性质:$\eset \subseteq A$ 对任意集合 $A$ 成立。
	
	集合关系:$A \subseteqq B$,$B \supsetneq C$。
	
	% 集合包含
	$A \subsetneq B$,$B \supseteqq A$
	
	% 全序关系
	$a \torder b$($a \leq b$),$x \tordereq y$($x \preceq y$)
	
	% 偏序关系
	$u \porder v$($u \preceq v$),$s \pordereq t$($s \preccurlyeq t$)
	
	全序关系:$a \torder b$,偏序关系:$x \porder y$。
	
	有界线性算子:$\op{T}{X}{Y} \in \bounded(X,Y)$。
\end{tcblisting}

\begin{tcblisting}{}
	\begin{align*}
		\BetaFunc{3}{4} &= \BetaGammaRel{3}{4} 
		= \frac{\GammaFunc{3}\GammaFunc{4}}{\GammaFunc{7}} 
		= \frac{2! \cdot 3!}{6!} 
		= \frac{12}{720} 
		= \frac{1}{60} \quad [[1]][[10]]
	\end{align*}
\end{tcblisting}

\begin{tcblisting}{}
	% 示例
	\begin{align*}
		\text{拉普拉斯变换:} &\quad \LT{f(t)}{s} = \int_{0}^{\infty} f(t)e^{-st}dt \quad [[1]][[4]] \\
		\text{拉普拉斯反变换:} &\quad \ILT{F(s)}{t} = \frac{1}{2\pi i}\lim_{T\to\infty}\int_{\gamma - iT}^{\gamma + iT} F(s)e^{st}ds \quad [[1]][[9]] \\
		\text{应用示例:} &\quad \LT{\sin(\omega t)}{s} = \frac{\omega}{s^2 + \omega^2} \quad [[4]][[5]]
	\end{align*}
\end{tcblisting}

\begin{tcblisting}{}
	% 示例
	\begin{align*}
		\text{傅里叶变换:} &\quad \FT{f(t)}{\omega} = \int_{-\infty}^{\infty} f(t)e^{-i\omega t}dt \quad  \\
		\text{傅里叶反变换:} &\quad \IFT{F(\omega)}{t} = \frac{1}{2\pi}\int_{-\infty}^{\infty} F(\omega)e^{i\omega t}d\omega   \quad \\
		\text{应用示例:} &\quad \FT{e^{-at}u(t)}{\omega} = \frac{1}{a + i\omega} \quad (a > 0) 
	\end{align*}
\end{tcblisting}

\begin{tcblisting}{}
	% 使用拉普拉斯变换求解微分方程 y'' + 3y' + 2y = u(t)
	\begin{align*}
		\LT{y'' + 3y' + 2y}{s} &= \LT{u(t)}{s} \\
		\Rightarrow s^2Y(s) - sy(0) - y'(0) + 3sY(s) - 3y(0) + 2Y(s) &= \frac{1}{s} \quad [[1]][[5]]
	\end{align*}
\end{tcblisting}


\begin{tcblisting}{sidebyside}
	几何级数公式:
	\[
	\Sum{k=0}{\infty} x^k = \frac{1}{1-x} \quad (|x| < 1)
	\]
\end{tcblisting}
\begin{tcblisting}{sidebyside}
	二重积分:
	\[
	\Dint{\Omega}{x^2 + y^2}
	\]
\end{tcblisting}
\begin{tcblisting}{sidebyside}
	三重积分:
	\[
	\Tint{V}{xyz}
	\]
\end{tcblisting}
\begin{tcblisting}{sidebyside}
	环路积分示例:
	\[
	\oint_{C} \mathbf{F} \cdot d\mathbf{r}
	\]
\end{tcblisting}
\begin{tcblisting}{sidebyside}
	格林公式应用:
	\[
	\greens{P}{Q}
	\]
	
	% 花体符号
	\[
	\mathcal{F}\{f(t)\} = \int_{-\infty}^{\infty} f(t) e^{-i\omega t} dt  % 傅里叶变换
	\]
	
	% 希腊字母
	\[
	\lim_{\Delta x \to 0} \frac{\Delta y}{\Delta x} = \frac{dy}{dx}  % 导数定义
	\]
	
	% 变体符号
	\[
	\text{角度:} \theta = \varphi = 45^\circ, \quad \text{误差:} \epsilon \ll \varepsilon
	\]
\end{tcblisting}


\begin{kuohao}
	\item First item in parentheses.1111
	\item Second item in parentheses.
\end{kuohao}

\begin{change}
	\item First change.
	\item Second change.
\end{change}

\begin{change2}
	\item First change in a box.
	\item Second change in a box.
\end{change2}

\begin{changecircredtwo}
	\item First change.
	\item Second change.
\end{changecircredtwo}

\begin{changecicred}
	\item First change.
	\item Second change.
\end{changecicred}

% 基础盒子
\begin{basebox}{}{这是一个标题}
	这是默认样式的盒子内容,支持自动换行和数学公式:$E=mc^2$。
\end{basebox}

% 自定义颜色和宽度
\begin{basebox}{colback=blue!5!white, colframe=blue!50!black}{蓝色主题盒子}
	内容区域背景色为浅蓝,边框为深蓝色。
\end{basebox}


\begin{leftimage}[0.4\textwidth]{example-image-a}
	This is the description for the left image.
\end{leftimage}

\begin{leftimage}[0.4\textwidth]{example-image-a}{示例图片 A}
	这是图片 A 的说明文字。
\end{leftimage}

\begin{lefttwoimage}[0.4\textwidth]{example-image-c}{示例图片 c}
	图c
\end{lefttwoimage}
\begin{leftimagewithcaption}[0.4\textwidth]{example-image-b}{示例图片 B}{fig:example-b}
	这是图片 B 的说明文字,并带有标题和标签。
\end{leftimagewithcaption}



\pseudocodetwo{
	\KwIn{$x$, $y$}
	\KwOut{$z$}
	$z \gets x + y$ \;
	\KwRet{$z$}
}

\pseudocodeinabox{
	\KwIn{$x$, $y$}
	\KwOut{$z$}
	$z \gets x + y$ \;
	\KwRet{$z$}
}


% 测试 change2 环境
\begin{change2}
	\item 第一个条目。
	\item 第二个条目。
	\item 第三个条目。
\end{change2}

% 测试 kuohao 环境
\begin{kuohao}
	\item 这是第一个条目。
	\item 这是第二个条目。
	\item 这是第三个条目。
\end{kuohao}
\ascboxB*[C]{This is my title}

\begin{proof}
	This is the proof.
\end{proof}
\solc
\begin{equation*}
	\frac{GMm}{r^2} = m \omega^2 r \RA \frac{GMm}{r^2} = m\left(\frac{2\pi}{T}\right)^2 r \RA r^3
	= \frac{GMT^2}{4\pi^2}
\end{equation*}

$A \implies B$ 表示 $A$ 推出 $B$。

- $A \iff B$ 表示 $A$ 等价于 $B$。

- $A \limplies B$ 表示 $B$ 推出 $A$(左推出)。


在1公式环境中:
\[
P \implies Q, \quad P \iff Q, \quad P \limplies Q
\]


在2公式环境中:
\[
P \impliestwo Q, \quad P \ifftwo Q, \quad P \limpliestwo Q
\]

在正文中也可以直接使用:\\
如果 A \impliestwo B,那么 B \limpliestwo A。等价关系可以写为 A \ifftwo B。


这是一个数学推导的示例:

\[
\be \text{上确界和下确界}
\]

\[
\so \text{结论成立}
\]

在正文中也可以使用:\be 上确界和下确界,\so 结论成立。

