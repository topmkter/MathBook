\usepackage{fontspec}
\usepackage{xeCJK}

%\setmainfont{毛泽东字体}[Path = ./]
%\setCJKmainfont[Path = ./]{毛泽东字体}   %% 这个是全局设置为 毛主席的字体

\setCJKfamilyfont{maoti}[Path=./fonts/, Extension=.ttf, BoldFont=*]{maoti}
\newcommand*{\maoti}{\CJKfamily{maoti}} %%% 毛主席字体

\setCJKfamilyfont{zhen}[Path=./fonts/, Extension=.ttf, BoldFont=*]{huawenzhengkai}
\newcommand*{\zhen}{\CJKfamily{zhen}} % 华文正楷

\setCJKfamilyfont{boyang}[Path=./fonts/, Extension=.ttf, BoldFont=*]{boyang}
\newcommand*{\boyang}{\CJKfamily{boyang}}  % 博洋草书3500

\setCJKfamilyfont{deng}[Path=./fonts/, Extension=.ttf, BoldFont=*]{deng}
\newcommand*{\deng}{\CJKfamily{deng}} % 邓小平字体

\setCJKfamilyfont{shou}[Path=./fonts/, Extension=.ttf, BoldFont=*]{shou}
\newcommand*{\shou}{\CJKfamily{shou}}  % 字魂瘦金体(商用需授权)


%% 不起作用的原因是:专门用于中日韩等双字节字体,而AlexBrush/Sudestada是纯西文字体
%\setCJKfamilyfont{alex}[Path=./fonts/, Extension=.ttf, BoldFont=*]{alex}
%\newcommand*{\alex}{\CJKfamily{alex}}  % AlexBrush-7XGA
%\setCJKfamilyfont{qianming}[Path=./fonts/, Extension=.ttf, BoldFont=*]{sun}
%\newcommand*{\qianming}{\CJKfamily{qianming}}  % Sudestada-Mine Cover.ttf 签名的用

% 西文字体定义
\newfontfamily\alexfont[Path=./fonts/, Extension=.ttf]{alex} 
\newcommand{\alex}{\alexfont}  % 签名草体

\newfontfamily\qianmingfont[Path=./fonts/, Extension=.ttf]{sun} 
\newcommand{\qianming}{\qianmingfont}  % 签名草体

\newfontfamily\Liorafont[Path=./fonts/, Extension=.ttf]{LiorahBT} 
\newcommand{\Liora}{\Liorafont} % 签名草体

\newfontfamily\timefont[Path=./fonts/, Extension=.ttf]{times-new-yorker-1} 
\newcommand{\Timef}{\timefont} % 墨水

\newfontfamily\InterLifont[Path=./fonts/, Extension=.ttf]{Inter-Light} 
\newcommand{\interLi}{\InterLifont} % 正常

\newfontfamily\Interfont[Path=./fonts/, Extension=.ttf]{InterVariable-Italic} 
\newcommand{\interxie}{\Interfont} % 斜体

\newfontfamily\InterBofont[Path=./fonts/, Extension=.ttf]{Inter-Bold} 
\newcommand{\inter}{\InterBofont}  %% 粗体 

\newfontfamily\Maplefont[Path=./fonts/, Extension=.ttf]{MapleMono-NF-CN-MediumItalic} 
\newcommand{\maple}{\Maplefont}  %% maple 字体

\newfontfamily\ZifferaVeneta[Path=./fonts/, Extension=.ttf]{ZifferaVeneta} 
\newcommand{\ziff}{\ZifferaVeneta}  %% 鬼画符字体

\newfontfamily\yidakixie[Path=./fonts/, Extension=.ttf]{yidakixie} 
\newcommand{\yidali}{\yidakixie}  %% 鬼画符字体

%% 还需要设计一些艺术字体进来
