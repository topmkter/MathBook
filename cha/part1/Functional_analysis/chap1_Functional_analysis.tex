\chapter{\maoti 泛函分析}
\vspace{20.4pt}

《泛函分析》是“更广泛、更一般化的”《数学分析》,将分析中的具体问题抽象到一种更加纯粹的代数、拓扑的形式中加以研究,综合运用分析、代数、几何的观点与方法,研究无限维空间上的函数、算子和极限理论,解决分析学中的问题.
\begin{center}
	\textcolor[RGB]{255, 0, 0}{\faHeart}在学习中要敢于做减法,减去前人已经解决的部分,看看还有那些问题没有解决,需要我们去探索解决。\textcolor[RGB]{255, 0, 0}{\faHeart}
\end{center}
\rightline{\zhen ---华罗庚}
\vspace{-5pt}
\begin{center}
	\pgfornament[width=0.36\linewidth,color=lsp]{88}
\end{center}
 
 我们认为要真正理解泛函分析中的一些重要的概念和理论,灵活运用这一强有力的工具,其唯一的途径就是深入了解它们的来源和背景,注重研究一些重要的、一般性定理的深刻的、具体的含义.不然的话,如果只
是从概念到概念,纯形式地理解抽象定理的推演,那么学习泛函分析的结果只能是“\textbf{如宝山而空返,一无所获.}”
\rightline{\zhen ---张恭庆院士} 


\section{\deng 距离空间}

 \subsection{\zhen 距离空间的基本概念}
 
 \faSendO  只需要在集合$X$ \X 上能定义衡量接近的程度计算方式,就可以基于这个方式来定义极限,从而有完备、紧性、微分,积分,级数...... 
 	\subsubsection{基本概念}
 	\begin{definition}[距离空间] \label{def:001}
 		设$X$是任一非空集合,$\forall x,y \in \X $,$\exists d(x,y) \in \R $ 并且满足
 	\begin{kuohaotwo}
 		\item 非负性:$d(x,y) \ge 0, \Leftrightarrow x = y \to d(x,y)=0$ ,
 		\item 对称性: $\forall x,y \in \X, d(x,y) = d(y,x) $
 		\item 三角不等性(可由1,2推出):$\forall x,y,z \in \X, d(x,y) \le d(x,z) + d(z,y)$
 	\end{kuohaotwo}
称$d(x,y)$ 为$\X$定义的一个距离,定义了距离$d$的这样一个集合记作$(\X,d)$ ,也叫距离空间,不要求是线性空间。
 	\end{definition} 

\begin{anymark}[总结~一些常见的距离空间]
	要想证明一个集合是距离空间,只需要证明定义的$d$符合上述三个性质:\textbf{非负正定、正齐次性、三角不等式}
	\begin{kuohao}
		\item  n维度欧氏空间,$\R^n = \set{(\epsilon_1,\epsilon_2 ,...\epsilon_n),\epsilon_i \in \K}$ ,要证明这个需要证明Cauchy不等式(Switch不等式)。n维复数空间$\C^n$也可以,需要取模就行。
		\item $ \textbf{C[a,b]} ,\text{其中},d(x,y) =  \max_{a \leq t \leq b} |x(t) - y(t)|,\forall x(t),y(t) \in \Cont$
		\item \textbf{空间 $s$} : 实数列$ \xi_k$ 的全体。设$x = \{\xi_k\},y = \{\eta_k\}$ 是两个实数列,定义
		$d(x, y) = \sum_{k=1}^{\infty} \frac{1}{2^k} \cdot \frac{|\xi_k - \eta_k|}{1 + |\xi_k - \eta_k|}.$
		上式右边的 $\frac{1}{2^k}$ 是一个收敛因子,保证级数收敛
		\item \textbf{空间 $S$} :	 $E$ 是一个 有限Lebesgue 可测集,,$E$ \text{上几乎处处有穷的可测函数全体}, \text{其中凡几乎处处相等的函数看成是同一元},$d(x,y)= \int_E \frac{|x(t) - y(t)|}{1 + |x(t) - y(t)|} dt.$
		 \item 离散空间D : 
		 $d(x,y) = \begin{cases}
		 	0 & x= y\\
		 	1 & x \ne y
		 \end{cases}$
		 \item 测度空间\Lpe{p} 上的$p$ 次幂L-可积函数空间\Lpe{p},$1\le p <\infty$,  简记为$L^p(\Omega)$ ; p次幂可和的数列空间$\lp{p},1 \le p <\infty ,\lp{p}=\set{\set{\epsilon_n},\sum_{n=1}^{\infty}\epsilon_n<\infty}$
		 \item 几乎处处有界可测函数空间$\Lpe{\infty}$,简记为$\Linf(\Omega)$。有界数列空间\linf
	\end{kuohao}
\end{anymark}

\begin{definition}[距离空间的收敛]
	设$(\X,d)$为距离空间,$\set{x_n}\in \X $为点列,若$\exists x_0 \in \X ,d(x_n,x_0)\to 0$,则称$\set{x_n}$依距离收敛到$x_0$
	收敛的性质和一般的数分定义一致,极限若存在,必唯一,且所有子列都收敛于同一个极限。
\end{definition} 
\begin{theorem}[] \label{theorem:001}
	$(X,d)$为距离空间,则有$|d(x,y)-d(x_1,y_1)| \le |d(x,x_1)|+|d(y,y_1)|  \; x,y,x_1,y_1 \in X$ \hfill
	
说明一点:$\textbf{当}x_n\to x_0 \; and \; y_n \to y_o ,\textbf{必有} d(x_n,y_n) \to d(x_0,y_0)$	
\end{theorem}
	
	
\begin{anymark}
	$C[a,b]$ 的收敛是函数列在$[a,b]$上的\textbf{一致收敛}
	
	空间$S$的收敛等价于函数列依测度收敛
	
	\textbf{离散空间中},${x_n}\to x_0 \Leftrightarrow {x_n} == {x_0} \; n>n_0$ 
\end{anymark}
\begin{definition}[连续映射,等距] \label{definition:002}
	\begin{kuohaotwo}
		\item 	$(X,d),(X_1,d_1),\; f:X\to X_1 \hspace{1mm} ; x_0 \in X,\forall \,\epsilon >0 , \exists \delta >0 ,d(x,x_0) \le \delta \Leftarrow \forall x\in X ,d_1(f(x),f(x_0)) <\epsilon $  称\(f\)在\(x_0\)连续,若\(f\)在\(X\)上任意一点都连续,则称 \(f\)在\(X\)上连续
		\item  等距映射 $\forall x,y \in X \; d_{1}(f(x),f(y)) =d(x,y)$
		一个等距映射一定是一个连续映射并且是一一映上的,但不一定是映上的
	\end{kuohaotwo}
	
	要称俩空间等距,需要俩空间的距离的函数之间存在有一个\textbf{映上的}等距映射
\end{definition}

 	\subsubsection{点集}
\begin{definition*}[点的定义]

\[
\lim_{n\to\infty}\frac{1}{n}=0
\]

\[
\lim_x\left(1+\frac{1}{x}\right)^x = e
\]
\end{definition*}
 	\subsubsection{完备化---距离空间}
 	
 \subsection{\zhen 开集、闭集及连续映射}
 
 
 
 \subsection{\zhen 稠密与可分}
 	
 \subsection{\zhen 完备性,barie 纲定理}
 
 \subsection{\zhen 列紧与紧}
 
 \subsection{ \zhen Banach 压缩映射原理}