\chapter{\maoti 数学分析1--极限}
\vspace{1cm}
\section{\zhen 基础入门--实数集与函数}
将实数进行划分(戴德金定理等实数基本理论),可以分为无理数(不可数)与有理数(可数)的部分,有理数可以表示为俩个不可约整数比值,也可以用有限十进制小数或者无限十进制循环小数表示,无理数则用无限不循环十进制小数表示

\begin{definition}[小数的逼近] \label{def:part10001}
	在正实数中,取$x = a_0.a_1 a_2 a_3 \cdots$
	定义 \textbf{n位不足近似}:$X_n = a_0.a_1 a_2 a_3 \cdots a_n $ \\
	定义 \textbf{n位剩余近似 }: $\overline{x_n} = a_0.a_1 a_2 a_3 \cdots a_n + \frac{1}{10^n}=x_n+ \frac{1}{10^n}$ \\
	
	对应于负实数则正好相反 \\
\end{definition}

\begin{conclusion}[大小关系]
 $\forall x \in \R ,  x_{n} \le x_{n+1} \le x \le \overline{x_n+1} \le \overline{x_{n+1}}  $\\
 
 $\forall  x ,y \in \R , x>y \Leftrightarrow x_n > \overline{y_n} $
 
\end{conclusion}


\begin{conclusion}[实数性质]
	\begin{enumerate}
		\item 实数构成实数域,对普通加法和数量乘法运算封闭,满足结合律等
		\item 三歧性:$\forall x ,y \in \mathbb{R} , x >y , x < y , x = y $ 一定满足且只满足其中一项关系(良序性)
		\item 全序关系满足传递性
		\item 阿基米德性质 :$\forall b >a >0\in  \R , \exists n \in \N , na >b $
		\item 实数集	$\R$ 稠密,任何俩个不相等的实数之间必有另一个实数,既有有理数,又有无理数,有理数在无理数中稠密,$\R$ 可分。
		\item 	$\R$  与数轴上的点一一对应。
	\end{enumerate}

\end{conclusion}



\begin{definition}[有界定义] \label{def:part10002}
	有界集:既有上界,又有下界 ,$\exists M(L) >0 ,\forall x\in S, |x |<M(L)$\\
	
	上确界,$\forall x \in S ,x<\eta . \forall \epsilon >0 , \exists x_0 \in S ,x_0 > \eta - \epsilon $ \\
	
	下确界, $\forall x \in S ,x>\beta . \forall \epsilon >0 , \exists x_0 \in S ,x_0  <\beta + \epsilon $ \\
	
\end{definition}	

\begin{theorem}[确界定理] \label{theorem:0001}
	$S \ne \varnothing , \text{若S有上界,则S有上确界;若有下界,则必有下确界。}$
\end{theorem}

\begin{proof}
	
	$\because$ 上确界和下确界
	
	$\therefore$
	
	\so wj 
	
	\so kjds
	
	\be kjd
\end{proof}
\section{基础入门}
数据库

\section{基础入门}
asdjka

\section{基础入门}
AK就是
\section{基础入门}

啊神经病
	\subsection{shdj}
	ajk
\subsubsection{hjd}

\paragraph*{Paragraph P1} This is a basic unit of a group of sentences.

\subparagraph*{Sub Paragraph I} This a sub paragraph of P1. 
klsd 
swjkfn
jwebf
jksens
jsdbn
smfjbs
\subparagraph*{Sub Paragraph II} This a sub paragraph of P1. 
sdfa

ewf


we


eq

\subsubsection{Sub Sub Section I-I-II} 
This is another sub sub section.