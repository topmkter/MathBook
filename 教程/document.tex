\documentclass[12pt,a4paper]{ctexart}
\usepackage{amsmath,amssymb}
\usepackage{color}
\usepackage{geometry}
\usepackage{array}
\usepackage{multirow}
\geometry{margin=2.5cm}

% 定义常用数学符号和集合符号
\providecommand{\R}{\ensuremath{\mathbb{R}}} % 实数集
\providecommand{\N}{\ensuremath{\mathbb{N}}} % 自然数集
\providecommand{\Z}{\ensuremath{\mathbb{Z}}} % 整数集
\providecommand{\Q}{\ensuremath{\mathbb{Q}}} % 有理数集
\providecommand{\C}{\ensuremath{\mathbb{C}}} % 复数集
\providecommand{\F}{\ensuremath{\mathbb{F}}} % 一般域(Field)
\providecommand{\K}{\ensuremath{\mathbb{K}}} % 一般数域(Number Field)
\providecommand{\B}{\ensuremath{\mathbb{B}}}


% 定义快捷命令
\newcommand{\be}{\ensuremath{\;\because\;}}  % 快捷命令 \be 表示 $\because$
\newcommand{\so}{\ensuremath{\;\therefore\;}} % 快捷命令 \so 表示 $\therefore$

\newcommand{\GammaFunc}[1]{\Gamma(#1)}       % Gamma函数符号
\newcommand{\GammaDef}{\int_0^\infty t^{x-1}e^{-t}dt}  % Gamma积分定义
\newcommand{\GammaRec}[1]{\Gamma(#1+1) = #1\Gamma(#1)} % 递推关系
\newcommand{\BetaFunc}[2]{\mathrm{B}(#1,#2)}  % Beta函数符号
\newcommand{\BetaDef}{\int_0^1 t^{a-1}(1-t)^{b-1}dt}  % Beta积分定义
\newcommand{\BetaGammaRel}[2]{\frac{\GammaFunc{#1}\GammaFunc{#2}}{\GammaFunc{#1+#2}}} % Beta-Gamma关系
% 指数函数
\newcommand{\expf}[2]{#1^{#2}}  % a^x
\newcommand{\expe}[1]{\mathrm{e}^{#1}}  % e^x
\newcommand{\expp}[2]{#1^{\left( #2 \right)}}  % a^{(x)}

% 欧拉函数
\newcommand{\euler}[1]{\phi\left( #1 \right)}  % φ(n)

% 欧拉公式
\newcommand{\eulerformula}[1]{\mathrm{e}^{\mathrm{i}#1} = \cos#1 + \mathrm{i}\sin#1}

% 对数函数
\newcommand{\logbase}[2]{\log_{#1}\left( #2 \right)}  % log_a(x)
\newcommand{\lnx}[1]{\ln\left( #1 \right)}  % ln(x)

% 复数与三角函数
\newcommand{\complexexp}[1]{\mathrm{e}^{\mathrm{i}#1}}  % e^{iθ}
\newcommand{\trig}[3]{#1\left( #2 \right) = #3}  % 三角函数模板

% 定义概率和期望符号
\providecommand{\E}{\ensuremath{\mathbb{E}}} % 数学期望
\providecommand{\P}{\ensuremath{\mathbb{P}}} % 概率测度

% 定义常用数学运算符
\newcommand{\abs}[1]{\left| #1 \right|} % 绝对值
\newcommand{\norm}[1]{\left\| #1 \right\|} % 范数
\newcommand{\inner}[2]{\langle #1, #2 \rangle} % 内积
\newcommand{\set}[1]{\left\{ #1 \right\}} % 集合表示
\newcommand{\paren}[1]{\left( #1 \right)} % 圆括号自动调整大小



% 定义逻辑符号
\providecommand{\implies}{\ensuremath{\Rightarrow}}       % 推出符号(右箭头)
\providecommand{\iff}{\ensuremath{\Leftrightarrow}}      % 等价符号(双向箭头)
\providecommand{\limplies}{\ensuremath{\Leftarrow}}      % 左推出符号(左箭头)

% 定义逻辑符号(带额外间距)
\providecommand{\impliestwo}{\ensuremath{\;\Rightarrow\;}}
\providecommand{\ifftwo}{\ensuremath{\;\Leftrightarrow\;}}
\providecommand{\limpliestwo}{\ensuremath{\;\Leftarrow\;}}
\newcommand{\dto}[2]{\overset{#1}{\rightarrow} #2}

\providecommand{\AND}{\wedge} % 逻辑与
\providecommand{\OR}{\vee} % 逻辑或
\providecommand{\notimplies}{\nRightarrow} % 不推出符号

% 定义矩阵相关符号
\newcommand{\mat}[1]{\mathbf{#1}} % 矩阵加粗表示
\newcommand{\vecb}[1]{\mathbf{#1}} % 向量加粗表示
\newcommand{\rank}[1]{\mathrm{rank}(#1)} % 矩阵的秩
\newcommand{\trace}[1]{\mathrm{tr}(#1)} % 矩阵的迹

% 定义其他常用符号
\newcommand{\degree}{^\circ} % 角度符号
\newcommand{\limit}[2]{\lim_{#1 \to #2}} % 极限符号
\newcommand{\argmax}{\operatorname*{arg\,max}} % 最大值点
\newcommand{\argmin}{\operatorname*{arg\,min}} % 最小值点

% 级数快捷命令
\newcommand{\Sum}[2]{\sum\limits_{#1}^{#2}}

% 多重积分快捷命令
\newcommand{\Dint}[2]{\iint_{#1} #2 \, dx dy}
\newcommand{\Tint}[2]{\iiint_{#1} #2 \, dx dy dz}

% 测度
\newcommand{\measure}[1]{\mu\left( #1 \right)}
\newcommand{\lebmeasure}{\mathcal{L}}

% 空集
\newcommand{\eset}{\emptyset}
\newcommand{\vempty}{\varnothing}

% === 集合包含关系(使用 \providecommand 避免冲突) ===
% 严格子集/包含(已通过 amssymb 定义,此处仅确保兼容)[[1]][[7]][[9]]
\providecommand{\subsetneq}{\subsetneq}   % A ⊊ B
\providecommand{\supsetneq}{\supsetneq}   % B ⊋ A
\providecommand{\subseteqq}{\subseteqq}   % A ⊆ B [[9]]
\providecommand{\supseteqq}{\supseteqq}   % B ⊇ A [[9]]

% === 全序关系(Total Order) ===
\newcommand{\torder}{\leq}        % 全序 ≤(标准符号)
\newcommand{\tordereq}{\preceq}   % 全序 ≼(扩展符号,如字典序)

% === 偏序关系(Partial Order) ===
\newcommand{\porder}{\preceq}     % 偏序 ≼(标准符号)
\newcommand{\pordereq}{\preccurlyeq} % 偏序 ≼(花体风格,如分量序)[[10]]

% 泛函分析算子
\newcommand{\op}[3]{\ensuremath{#1: #2 \to #3}}                  % 算子定义:T: X → Y
\newcommand{\bounded}{\ensuremath{\mathcal{B}}}                % 有界线性算子空间 [[7]]
\newcommand{\compact}{\ensuremath{\mathcal{K}}}                % 紧算子(Compact operator)[[2]]
\newcommand{\LP}[1]{\ensuremath{\mathrm{L}^{#1}}}              % L^p 空间(参数化版本)
\newcommand{\Linf}{\ensuremath{\mathrm{L}^\infty}}              % L^∞ 空间
\newcommand{\lp}[1]{\ensuremath{\ell^{#1}}}                    % l^p 空间(使用 \ell 符号)
\newcommand{\linf}{\ensuremath{\ell^\infty}}                   % l^∞ 空间
\providecommand{\Lpe}[1]{\ensuremath{L^{#1}(\Omega, S, m)}}  % L^p 空间 
\providecommand{\OSM}{\ensuremath{(\Omega, S, m)}}          % 测度空间三元组 
\providecommand{\Cab}{\ensuremath{C[a, b]}}               % 连续函数空间 
\providecommand{\Linf}{\ensuremath{L^\infty(\Omega, S, m)}} % L^∞ 空间 
% 微分与积分符号
\newcommand{\diff}{\ensuremath{\,\mathrm{d}}}                   % 微分符号
\newcommand{\pd}[2]{\ensuremath{\frac{\partial #1}{\partial #2}}} % 偏导数 [[8]]
\newcommand{\intab}[3]{\ensuremath{\int_{#1}^{#2} #3 \diff #1}} % 定积分

% 拓扑与空间
\newcommand{\Banach}{\ensuremath{\text{Banach}}}               % Banach 空间 [[4]]
\newcommand{\Haus}{\ensuremath{\text{Hausdorff}}}              % Hausdorff 空间
\newcommand{\Hilbert}{\ensuremath{\mathcal{H}}}                % Hilbert 空间 [[1]]

% 扩展:连续函数与对偶空间
\newcommand{\Cont}[1]{\ensuremath{\mathrm{C}(#1)}}             % 连续函数空间 C(X)
\newcommand{\Dual}[1]{\ensuremath{{#1}^*}}                     % 对偶空间 X*

% 格林公式
\newcommand{\greens}[2]{%
	\oint_{\partial D} \left( #1 \, dx + #2 \, dy \right) = 
	\iint_{D} \left( \pd{#2}{x} - \pd{#1}{y} \right) dx dy
}

% 傅里叶变换
\newcommand{\FT}[2]{\mathcal{F}\left\{#1\right\}(#2)}  
% 傅里叶反变换
\newcommand{\IFT}[2]{\mathcal{F}^{-1}\left\{#1\right\}(#2)}  

% 拉普拉斯变换
\newcommand{\LT}[2]{\mathcal{L}\left\{#1\right\}(#2)}  
% 拉普拉斯反变换
\newcommand{\ILT}[2]{\mathcal{L}^{-1}\left\{#1\right\}(#2)}  


% 定义文本格式化命令
\newcommand{\emphr}[1]{\textbf{\color{red}#1}} % 强调红色文本
\newcommand{\emphb}[1]{\textbf{#1}} % 强调加粗文本
\newcommand{\todo}[1]{\textbf{\color{blue}[TODO: #1]}} % 待办事项标记

% 定义方程组快捷命令
\newcommand{\eqgroup}[1]{%
	\begin{cases}
		#1
	\end{cases}
}

% 定义矩阵快捷命令
\newcommand{\pmat}[1]{%
	\begin{pmatrix}
		#1
	\end{pmatrix}
}
\newcommand{\bmat}[1]{%
	\begin{bmatrix}
		#1
	\end{bmatrix}
}
\newcommand{\matr}[1]{%
	\begin{matrix}
		#1
	\end{matrix}
}
% 定义增广矩阵命令(修复列数)
\newcommand{\augmat}[1]{%
	\left[\begin{array}{@{}cc|c@{}}  % 三列:前两列是普通列,第三列用竖线分隔
		#1
	\end{array}\right]
}

\begin{document}
	
	\title{LaTeX 命令完整测试文档}
	\author{自动生成}
	\date{\today}
	\maketitle
	
	\section{数学集合符号}
	\begin{tabular}{|l|l|l|l|l|}
		\hline
		\textbf{命令} & \textbf{输入代码} & \textbf{输出效果} & \textbf{命令} & \textbf{输入代码} \\
		\hline
		实数集 & \verb|\R| & $\R$ & 自然数集 & \verb|\N| \\
		\hline
		整数集 & \verb|\Z| & $\Z$ & 有理数集 & \verb|\Q| \\
		\hline
		复数集 & \verb|\C| & $\C$ & 域符号 & \verb|\F| \\
		\hline
		数域 & \verb|\K| & $\K$ & 布尔集 & \verb|\B| \\
		\hline
	\end{tabular}
	
	\section{常用数学符号}
	\begin{tabular}{|c|c|c|c|c|}
		\hline
		\textbf{功能} & \textbf{输入代码} & \textbf{输出效果} & \textbf{功能} & \textbf{输出效果} \\
		\hline
		因为 & \verb|\be| & $\be$ & 所以 & $\so$ \\
		\hline
		绝对值 & \verb|\abs{-x}| & $\abs{-x}$ & 范数 & $\norm{x}$ \\
		\hline
		内积 & \verb|\inner{u}{v}| & $\inner{u}{v}$ & 集合 & $\set{a,b,c}$ \\
		\hline
		希腊字母 & \verb|\degree| & $\degree$ & 极限 & $\limit{x}{a}$ \\
		\hline
	\end{tabular}
	
	\section{特殊函数与变换}
	\begin{tabular}{|c|c|c|c|}
		\hline
		\textbf{函数类型} & \textbf{输入代码} & \textbf{输出效果} & \textbf{定义式} \\
		\hline
		Gamma函数 & \verb|\GammaFunc{z}| & $\GammaFunc{z}$ & $\GammaDef$ \\
		\hline
		Beta函数 & \verb|\BetaFunc{a}{b}| & $\BetaFunc{a}{b}$ & $\BetaDef$ \\
		\hline
		欧拉函数 & \verb|\euler{n}| & $\euler{n}$ & $\phi(12)=4$ \\
		\hline
		傅里叶变换 & \verb|\FT{f}{\omega}| & $\FT{f}{\omega}$ & $\mathcal{F}\{f(t)\}(\omega)$ \\
		\hline
	\end{tabular}
	
	\section{指数与对数函数}
	\begin{tabular}{|c|c|c|c|}
		\hline
		功能 & 输入代码 & 输出效果 & 参数说明 \\
		\hline
		普通幂函数 & \verb|\expf{a}{x}| & $\expf{a}{x}$ & 底数/指数 \\
		\hline
		自然指数 & \verb|\expe{x}| & $\expe{x}$ & e的幂次 \\
		\hline
		括号幂次 & \verb|\expp{a}{x}| & $\expp{a}{x}$ & 带括号 \\
		\hline
		对数函数 & \verb|\logbase{2}{8}| & $\logbase{2}{8}$ & 底数/真数 \\
		\hline
		自然对数 & \verb|\lnx{x}| & $\lnx{x}$ & ln函数 \\
		\hline
	\end{tabular}
	
	\section{矩阵与线性代数}
	\begin{tabular}{|c|c|c|}
		\hline
		功能 & 输入代码 & 输出效果 \\
		\hline
		矩阵表示 & \verb|\pmat{a&b\\c&d}| & $\pmat{a&b\\c&d}$ \\
		\hline
		增广矩阵 & \verb|\augmat{1&2&3\\4&5&6}| & $\augmat{1&2&3\\4&5&6}$ \\
		\hline
		矩阵秩 & \verb|\rank{A}| & $\rank{A}$ \\
		\hline
		矩阵迹 & \verb|\trace{A}| & $\trace{A}$ \\
		\hline
		向量表示 & \verb|\vecb{v}| & $\vecb{v}$ \\
		\hline
	\end{tabular}
	
	\section{积分与微分}
	\begin{tabular}{|c|c|c|}
		\hline
		功能 & 输入代码 & 输出效果 \\
		\hline
		偏导数 & \verb|\pd{f}{x}| & $\pd{f}{x}$ \\
		\hline
		定积分 & \verb|\intab{a}{b}{f}| & $\intab{a}{b}{f}$ \\
		\hline
		双重积分 & \verb|\Dint{D}{f}| & $\Dint{D}{f}$ \\
		\hline
		格林公式 & \verb|\greens{P}{Q}| & $\greens{P}{Q}$ \\
		\hline
	\end{tabular}
	
	\section{概率与统计}
	\begin{tabular}{|c|c|c|}
		\hline
		功能 & 输入代码 & 输出效果 \\
		\hline
		概率测度 & \verb|\P(A)| & $\P(A)$ \\
		\hline
		数学期望 & \verb|\E[X]| & $\E[X]$ \\
		\hline
		测度空间 & \verb|\OSM| & $\OSM$ \\
		\hline
	\end{tabular}
	
	\section{逻辑与集合}
	\begin{tabular}{|c|c|c|c|}
		\hline
		符号类型 & 输入代码 & 输出效果 & 符号类型 \\
		\hline
		蕴含 & \verb|\implies| & $\implies$ & 等价 $\iff$ \\
		\hline
		子集 & \verb|\subsetneq| & $\subsetneq$ & 并集 $\cup$ \\
		\hline
		空集 & \verb|\eset| & $\eset$ & 补集 $\setminus$ \\
		\hline
	\end{tabular}
	
	\section{空间与算子}
	\begin{tabular}{|c|c|c|}
		\hline
		概念 & 输入代码 & 输出效果 \\
		\hline
		Banach空间 & \verb|\Banach| & \Banach \\
		\hline
		Hilbert空间 & \verb|\Hilbert| & $\Hilbert$ \\
		\hline
		$L^p$空间 & \verb|\LP{p}| & $\LP{p}$ \\
		\hline
		有界算子 & \verb|\bounded| & $\bounded$ \\
		\hline
	\end{tabular}
	
	

	\section{文本格式化}
	\begin{tabular}{|c|c|c|}
		\hline
		功能 & 输入代码 & 输出效果 \\
		\hline
		红色强调 & \verb|\emphr{警告}| & \emphr{警告} \\
		\hline
		加粗强调 & \verb|\emphb{重点}| & \emphb{重点} \\
		\hline
		待办事项 & \verb|\todo{修改}| & \todo{修改} \\
		\hline
	\end{tabular}
	
\end{document}